\documentclass[]{article}
\usepackage{lmodern}
\usepackage{amssymb,amsmath}
\usepackage{ifxetex,ifluatex}
\usepackage{fixltx2e} % provides \textsubscript
\ifnum 0\ifxetex 1\fi\ifluatex 1\fi=0 % if pdftex
  \usepackage[T1]{fontenc}
  \usepackage[utf8]{inputenc}
\else % if luatex or xelatex
  \ifxetex
    \usepackage{mathspec}
  \else
    \usepackage{fontspec}
  \fi
  \defaultfontfeatures{Ligatures=TeX,Scale=MatchLowercase}
\fi
% use upquote if available, for straight quotes in verbatim environments
\IfFileExists{upquote.sty}{\usepackage{upquote}}{}
% use microtype if available
\IfFileExists{microtype.sty}{%
\usepackage{microtype}
\UseMicrotypeSet[protrusion]{basicmath} % disable protrusion for tt fonts
}{}
\usepackage[margin=1in]{geometry}
\usepackage{hyperref}
\hypersetup{unicode=true,
            pdftitle={Transformasi Data},
            pdfauthor={Muhammad Aswan Syahputra},
            pdfborder={0 0 0},
            breaklinks=true}
\urlstyle{same}  % don't use monospace font for urls
\usepackage{color}
\usepackage{fancyvrb}
\newcommand{\VerbBar}{|}
\newcommand{\VERB}{\Verb[commandchars=\\\{\}]}
\DefineVerbatimEnvironment{Highlighting}{Verbatim}{commandchars=\\\{\}}
% Add ',fontsize=\small' for more characters per line
\usepackage{framed}
\definecolor{shadecolor}{RGB}{248,248,248}
\newenvironment{Shaded}{\begin{snugshade}}{\end{snugshade}}
\newcommand{\AlertTok}[1]{\textcolor[rgb]{0.94,0.16,0.16}{#1}}
\newcommand{\AnnotationTok}[1]{\textcolor[rgb]{0.56,0.35,0.01}{\textbf{\textit{#1}}}}
\newcommand{\AttributeTok}[1]{\textcolor[rgb]{0.77,0.63,0.00}{#1}}
\newcommand{\BaseNTok}[1]{\textcolor[rgb]{0.00,0.00,0.81}{#1}}
\newcommand{\BuiltInTok}[1]{#1}
\newcommand{\CharTok}[1]{\textcolor[rgb]{0.31,0.60,0.02}{#1}}
\newcommand{\CommentTok}[1]{\textcolor[rgb]{0.56,0.35,0.01}{\textit{#1}}}
\newcommand{\CommentVarTok}[1]{\textcolor[rgb]{0.56,0.35,0.01}{\textbf{\textit{#1}}}}
\newcommand{\ConstantTok}[1]{\textcolor[rgb]{0.00,0.00,0.00}{#1}}
\newcommand{\ControlFlowTok}[1]{\textcolor[rgb]{0.13,0.29,0.53}{\textbf{#1}}}
\newcommand{\DataTypeTok}[1]{\textcolor[rgb]{0.13,0.29,0.53}{#1}}
\newcommand{\DecValTok}[1]{\textcolor[rgb]{0.00,0.00,0.81}{#1}}
\newcommand{\DocumentationTok}[1]{\textcolor[rgb]{0.56,0.35,0.01}{\textbf{\textit{#1}}}}
\newcommand{\ErrorTok}[1]{\textcolor[rgb]{0.64,0.00,0.00}{\textbf{#1}}}
\newcommand{\ExtensionTok}[1]{#1}
\newcommand{\FloatTok}[1]{\textcolor[rgb]{0.00,0.00,0.81}{#1}}
\newcommand{\FunctionTok}[1]{\textcolor[rgb]{0.00,0.00,0.00}{#1}}
\newcommand{\ImportTok}[1]{#1}
\newcommand{\InformationTok}[1]{\textcolor[rgb]{0.56,0.35,0.01}{\textbf{\textit{#1}}}}
\newcommand{\KeywordTok}[1]{\textcolor[rgb]{0.13,0.29,0.53}{\textbf{#1}}}
\newcommand{\NormalTok}[1]{#1}
\newcommand{\OperatorTok}[1]{\textcolor[rgb]{0.81,0.36,0.00}{\textbf{#1}}}
\newcommand{\OtherTok}[1]{\textcolor[rgb]{0.56,0.35,0.01}{#1}}
\newcommand{\PreprocessorTok}[1]{\textcolor[rgb]{0.56,0.35,0.01}{\textit{#1}}}
\newcommand{\RegionMarkerTok}[1]{#1}
\newcommand{\SpecialCharTok}[1]{\textcolor[rgb]{0.00,0.00,0.00}{#1}}
\newcommand{\SpecialStringTok}[1]{\textcolor[rgb]{0.31,0.60,0.02}{#1}}
\newcommand{\StringTok}[1]{\textcolor[rgb]{0.31,0.60,0.02}{#1}}
\newcommand{\VariableTok}[1]{\textcolor[rgb]{0.00,0.00,0.00}{#1}}
\newcommand{\VerbatimStringTok}[1]{\textcolor[rgb]{0.31,0.60,0.02}{#1}}
\newcommand{\WarningTok}[1]{\textcolor[rgb]{0.56,0.35,0.01}{\textbf{\textit{#1}}}}
\usepackage{graphicx,grffile}
\makeatletter
\def\maxwidth{\ifdim\Gin@nat@width>\linewidth\linewidth\else\Gin@nat@width\fi}
\def\maxheight{\ifdim\Gin@nat@height>\textheight\textheight\else\Gin@nat@height\fi}
\makeatother
% Scale images if necessary, so that they will not overflow the page
% margins by default, and it is still possible to overwrite the defaults
% using explicit options in \includegraphics[width, height, ...]{}
\setkeys{Gin}{width=\maxwidth,height=\maxheight,keepaspectratio}
\IfFileExists{parskip.sty}{%
\usepackage{parskip}
}{% else
\setlength{\parindent}{0pt}
\setlength{\parskip}{6pt plus 2pt minus 1pt}
}
\setlength{\emergencystretch}{3em}  % prevent overfull lines
\providecommand{\tightlist}{%
  \setlength{\itemsep}{0pt}\setlength{\parskip}{0pt}}
\setcounter{secnumdepth}{0}
% Redefines (sub)paragraphs to behave more like sections
\ifx\paragraph\undefined\else
\let\oldparagraph\paragraph
\renewcommand{\paragraph}[1]{\oldparagraph{#1}\mbox{}}
\fi
\ifx\subparagraph\undefined\else
\let\oldsubparagraph\subparagraph
\renewcommand{\subparagraph}[1]{\oldsubparagraph{#1}\mbox{}}
\fi

%%% Use protect on footnotes to avoid problems with footnotes in titles
\let\rmarkdownfootnote\footnote%
\def\footnote{\protect\rmarkdownfootnote}

%%% Change title format to be more compact
\usepackage{titling}

% Create subtitle command for use in maketitle
\providecommand{\subtitle}[1]{
  \posttitle{
    \begin{center}\large#1\end{center}
    }
}

\setlength{\droptitle}{-2em}

  \title{Transformasi Data}
    \pretitle{\vspace{\droptitle}\centering\huge}
  \posttitle{\par}
    \author{Muhammad Aswan Syahputra}
    \preauthor{\centering\large\emph}
  \postauthor{\par}
      \predate{\centering\large\emph}
  \postdate{\par}
    \date{4/9/2019}


\begin{document}
\maketitle

{
\setcounter{tocdepth}{2}
\tableofcontents
}
\hypertarget{non-tidy-menjadi-tidy-dataset}{%
\subsection{Non-tidy menjadi Tidy
dataset}\label{non-tidy-menjadi-tidy-dataset}}

Anda akan menggunakan fungsi \texttt{spread} dari paket \texttt{tidyr}
untuk mengubah memperbaiki dataset `table2' (juga dari paket
\texttt{tidyr}). Aktifkanlah paket \texttt{tidyr}, lihat dataset
`table2'. Apakah yang membuat dataset tersebut \emph{non-tidy} data?
Bukalah dokumentasi fungsi \texttt{spread} dengan menjalankan
\texttt{?nama\_fungsi} atau \texttt{help(nama\_fungsi)}!

\#install package

\begin{Shaded}
\begin{Highlighting}[]
\CommentTok{#install.packages("tidyr")}
\CommentTok{#install.packages("skimr")}
\end{Highlighting}
\end{Shaded}

\#Aktifkanlah paket \texttt{tidyr}

\begin{Shaded}
\begin{Highlighting}[]
\CommentTok{#___(tidyr)}
\KeywordTok{library}\NormalTok{(tidyr)}
\end{Highlighting}
\end{Shaded}

\begin{verbatim}
## Warning: package 'tidyr' was built under R version 3.5.3
\end{verbatim}

\begin{Shaded}
\begin{Highlighting}[]
\NormalTok{table2}
\end{Highlighting}
\end{Shaded}

\begin{verbatim}
## # A tibble: 12 x 4
##    country      year type            count
##    <chr>       <int> <chr>           <int>
##  1 Afghanistan  1999 cases             745
##  2 Afghanistan  1999 population   19987071
##  3 Afghanistan  2000 cases            2666
##  4 Afghanistan  2000 population   20595360
##  5 Brazil       1999 cases           37737
##  6 Brazil       1999 population  172006362
##  7 Brazil       2000 cases           80488
##  8 Brazil       2000 population  174504898
##  9 China        1999 cases          212258
## 10 China        1999 population 1272915272
## 11 China        2000 cases          213766
## 12 China        2000 population 1280428583
\end{verbatim}

Catatan: tabel2 tidak tidy karena cases dan population dijadikan value,
seharusnya kolom.

Dataset `table2' dapat diperbaiki dengan menjalankan kode berikut:

\begin{Shaded}
\begin{Highlighting}[]
\NormalTok{table2_tidy <-}\StringTok{ }\KeywordTok{spread}\NormalTok{(table2, }\DataTypeTok{key =} \StringTok{"type"}\NormalTok{, }\DataTypeTok{value =} \StringTok{"count"}\NormalTok{)}
\CommentTok{#spread vs gahter. spread harus tau nama kolomnya. gather tidak harus tau nama kolomnya}
\CommentTok{#key kolom yang akan diubah menjadi nama kolom}
\CommentTok{#}
\CommentTok{#ubah key adalah kolom dari type, value adalah value dari count}
\NormalTok{table2_tidy}
\end{Highlighting}
\end{Shaded}

\begin{verbatim}
## # A tibble: 6 x 4
##   country      year  cases population
##   <chr>       <int>  <int>      <int>
## 1 Afghanistan  1999    745   19987071
## 2 Afghanistan  2000   2666   20595360
## 3 Brazil       1999  37737  172006362
## 4 Brazil       2000  80488  174504898
## 5 China        1999 212258 1272915272
## 6 China        2000 213766 1280428583
\end{verbatim}

Selanjutnya Anda juga akan memperbaiki dataset `table4a'. Cetaklah
dataset tersebut dan dapatkah Anda menyebutkan alasan mengapa dataset
tersebut tidak \emph{tidy} dan \emph{tame}?

\begin{Shaded}
\begin{Highlighting}[]
\NormalTok{table4a }\CommentTok{# cetak dataset table4a}
\end{Highlighting}
\end{Shaded}

\begin{verbatim}
## # A tibble: 3 x 3
##   country     `1999` `2000`
## * <chr>        <int>  <int>
## 1 Afghanistan    745   2666
## 2 Brazil       37737  80488
## 3 China       212258 213766
\end{verbatim}

catatan: tabel4a tidak tidy karena ada data observasi 1999 2000, malah
jadi kolom

Dataset `table4a' dapat diperbaiki dengan menggunakan fungsi
\texttt{gather} dari \texttt{tidyr}. Anda dapat mempelajari fungsi
tersebut dengan menjalankan \texttt{?gather}.

\begin{Shaded}
\begin{Highlighting}[]
\NormalTok{?gather }\CommentTok{#menampilkan help diawali dengan ?}
\end{Highlighting}
\end{Shaded}

\begin{verbatim}
## starting httpd help server ... done
\end{verbatim}

\begin{Shaded}
\begin{Highlighting}[]
\CommentTok{#gather mengubah kolom dimana saat kita ngga perlu tahu nama kolomnya, cukup gunakan index. perlu buat kolom baru}

\KeywordTok{gather}\NormalTok{(table4a, }\DataTypeTok{key =} \StringTok{"year"}\NormalTok{, }\DataTypeTok{value =} \StringTok{"case"}\NormalTok{, }\DecValTok{2}\OperatorTok{:}\DecValTok{3}\NormalTok{) }\CommentTok{#pakai index}
\end{Highlighting}
\end{Shaded}

\begin{verbatim}
## # A tibble: 6 x 3
##   country     year    case
##   <chr>       <chr>  <int>
## 1 Afghanistan 1999     745
## 2 Brazil      1999   37737
## 3 China       1999  212258
## 4 Afghanistan 2000    2666
## 5 Brazil      2000   80488
## 6 China       2000  213766
\end{verbatim}

\begin{Shaded}
\begin{Highlighting}[]
\KeywordTok{gather}\NormalTok{(table4a, }\DataTypeTok{key =} \StringTok{"year"}\NormalTok{, }\DataTypeTok{value =} \StringTok{"case"}\NormalTok{, }\StringTok{`}\DataTypeTok{1999}\StringTok{`}\OperatorTok{:}\StringTok{`}\DataTypeTok{2000}\StringTok{`}\NormalTok{) }\CommentTok{#atau pakai nama kolom. catatan: kalo pakai angka harus pakai backtick}
\end{Highlighting}
\end{Shaded}

\begin{verbatim}
## # A tibble: 6 x 3
##   country     year    case
##   <chr>       <chr>  <int>
## 1 Afghanistan 1999     745
## 2 Brazil      1999   37737
## 3 China       1999  212258
## 4 Afghanistan 2000    2666
## 5 Brazil      2000   80488
## 6 China       2000  213766
\end{verbatim}

\begin{Shaded}
\begin{Highlighting}[]
\KeywordTok{gather}\NormalTok{(table4a, }\DataTypeTok{key =} \StringTok{"year"}\NormalTok{, }\DataTypeTok{value =} \StringTok{"case"}\NormalTok{, }\KeywordTok{c}\NormalTok{(}\DecValTok{2}\NormalTok{,}\DecValTok{3}\NormalTok{)) }\CommentTok{#atau c jika tidak berurutan}
\end{Highlighting}
\end{Shaded}

\begin{verbatim}
## # A tibble: 6 x 3
##   country     year    case
##   <chr>       <chr>  <int>
## 1 Afghanistan 1999     745
## 2 Brazil      1999   37737
## 3 China       1999  212258
## 4 Afghanistan 2000    2666
## 5 Brazil      2000   80488
## 6 China       2000  213766
\end{verbatim}

\begin{Shaded}
\begin{Highlighting}[]
\CommentTok{#cara 1 pakai pipe}
\NormalTok{table4a_tidy <-}\StringTok{ }\NormalTok{table4a }\OperatorTok\StringTok{ }
\StringTok{  }\KeywordTok{gather}\NormalTok{(}\DataTypeTok{key =} \StringTok{"year"}\NormalTok{, }\DataTypeTok{value =} \StringTok{"kasus"}\NormalTok{, }\DecValTok{2}\OperatorTok{:}\DecValTok{3}\NormalTok{) }\CommentTok{# menggunakan tidyverse syntax, pipe %>%}
\NormalTok{table4a_tidy}
\end{Highlighting}
\end{Shaded}

\begin{verbatim}
## # A tibble: 6 x 3
##   country     year   kasus
##   <chr>       <chr>  <int>
## 1 Afghanistan 1999     745
## 2 Brazil      1999   37737
## 3 China       1999  212258
## 4 Afghanistan 2000    2666
## 5 Brazil      2000   80488
## 6 China       2000  213766
\end{verbatim}

\begin{Shaded}
\begin{Highlighting}[]
\CommentTok{#cara 2 pakai intermediate object}
\NormalTok{table4a_tidy <-}\StringTok{ }\KeywordTok{gather}\NormalTok{(table4a, }\DataTypeTok{key =} \StringTok{"year"}\NormalTok{, }\DataTypeTok{value =} \StringTok{"kasus"}\NormalTok{, }\DecValTok{2}\OperatorTok{:}\DecValTok{3}\NormalTok{) }
\NormalTok{table4a_tidy}
\end{Highlighting}
\end{Shaded}

\begin{verbatim}
## # A tibble: 6 x 3
##   country     year   kasus
##   <chr>       <chr>  <int>
## 1 Afghanistan 1999     745
## 2 Brazil      1999   37737
## 3 China       1999  212258
## 4 Afghanistan 2000    2666
## 5 Brazil      2000   80488
## 6 China       2000  213766
\end{verbatim}

Silakan lakukan hal serupa pada dataset `table4b' namun dengan
menggunakan ``population'' sebagai isian argumen \texttt{value}.
Tuliskan juga dengan menggunakan \emph{tidyverse syntax} dan simpan
obyek tersebut dengan nama `table4b\_tidy'!

\begin{Shaded}
\begin{Highlighting}[]
\NormalTok{table4b}
\end{Highlighting}
\end{Shaded}

\begin{verbatim}
## # A tibble: 3 x 3
##   country         `1999`     `2000`
## * <chr>            <int>      <int>
## 1 Afghanistan   19987071   20595360
## 2 Brazil       172006362  174504898
## 3 China       1272915272 1280428583
\end{verbatim}

\begin{Shaded}
\begin{Highlighting}[]
\KeywordTok{str}\NormalTok{(table4b)}
\end{Highlighting}
\end{Shaded}

\begin{verbatim}
## Classes 'tbl_df', 'tbl' and 'data.frame':    3 obs. of  3 variables:
##  $ country: chr  "Afghanistan" "Brazil" "China"
##  $ 1999   : int  19987071 172006362 1272915272
##  $ 2000   : int  20595360 174504898 1280428583
\end{verbatim}

\begin{Shaded}
\begin{Highlighting}[]
\CommentTok{#cara 1 pakai pipe}
\NormalTok{table4b_tidy <-}\StringTok{ }\NormalTok{table4b }\OperatorTok
\StringTok{    }\KeywordTok{gather}\NormalTok{(}\DataTypeTok{key =}\StringTok{"year"}\NormalTok{, }\DataTypeTok{value =}\StringTok{"population"}\NormalTok{, }\DecValTok{2}\OperatorTok{:}\DecValTok{3}\NormalTok{)}
\CommentTok{#cara 2 pakai Intermediate object}
\NormalTok{table4b_tidy <-}\StringTok{ }\KeywordTok{gather}\NormalTok{(table4b, }\DataTypeTok{key =}\StringTok{"year"}\NormalTok{, }\DataTypeTok{value =}\StringTok{"population"}\NormalTok{, }\DecValTok{2}\OperatorTok{:}\DecValTok{3}\NormalTok{)}
\NormalTok{table4b_tidy}
\end{Highlighting}
\end{Shaded}

\begin{verbatim}
## # A tibble: 6 x 3
##   country     year  population
##   <chr>       <chr>      <int>
## 1 Afghanistan 1999    19987071
## 2 Brazil      1999   172006362
## 3 China       1999  1272915272
## 4 Afghanistan 2000    20595360
## 5 Brazil      2000   174504898
## 6 China       2000  1280428583
\end{verbatim}

Dataset `table4a\_tidy' dan `table4b\_tidy' tersebut dapat digabungkan
menjadi satu dataset. Hal tersebut dapat dilakukan dengan menggunakan
fungsi \texttt{left\_join} dari paket \texttt{dplyr} seperti contoh
berikut:

\begin{Shaded}
\begin{Highlighting}[]
\KeywordTok{library}\NormalTok{(dplyr) }\CommentTok{# mengaktifkan paket dplyr}
\end{Highlighting}
\end{Shaded}

\begin{verbatim}
## Warning: package 'dplyr' was built under R version 3.5.3
\end{verbatim}

\begin{verbatim}
## 
## Attaching package: 'dplyr'
\end{verbatim}

\begin{verbatim}
## The following objects are masked from 'package:stats':
## 
##     filter, lag
\end{verbatim}

\begin{verbatim}
## The following objects are masked from 'package:base':
## 
##     intersect, setdiff, setequal, union
\end{verbatim}

\begin{Shaded}
\begin{Highlighting}[]
\NormalTok{table4a_tidy}
\end{Highlighting}
\end{Shaded}

\begin{verbatim}
## # A tibble: 6 x 3
##   country     year   kasus
##   <chr>       <chr>  <int>
## 1 Afghanistan 1999     745
## 2 Brazil      1999   37737
## 3 China       1999  212258
## 4 Afghanistan 2000    2666
## 5 Brazil      2000   80488
## 6 China       2000  213766
\end{verbatim}

\begin{Shaded}
\begin{Highlighting}[]
\NormalTok{table4b_tidy}
\end{Highlighting}
\end{Shaded}

\begin{verbatim}
## # A tibble: 6 x 3
##   country     year  population
##   <chr>       <chr>      <int>
## 1 Afghanistan 1999    19987071
## 2 Brazil      1999   172006362
## 3 China       1999  1272915272
## 4 Afghanistan 2000    20595360
## 5 Brazil      2000   174504898
## 6 China       2000  1280428583
\end{verbatim}

\begin{Shaded}
\begin{Highlighting}[]
\NormalTok{mydata <-}\StringTok{ }\KeywordTok{left_join}\NormalTok{(table4a_tidy, table4b_tidy, }\DataTypeTok{by =} \KeywordTok{c}\NormalTok{(}\StringTok{"country"}\NormalTok{, }\StringTok{"year"}\NormalTok{))}
\NormalTok{mydata}
\end{Highlighting}
\end{Shaded}

\begin{verbatim}
## # A tibble: 6 x 4
##   country     year   kasus population
##   <chr>       <chr>  <int>      <int>
## 1 Afghanistan 1999     745   19987071
## 2 Brazil      1999   37737  172006362
## 3 China       1999  212258 1272915272
## 4 Afghanistan 2000    2666   20595360
## 5 Brazil      2000   80488  174504898
## 6 China       2000  213766 1280428583
\end{verbatim}

\begin{Shaded}
\begin{Highlighting}[]
\CommentTok{#mengurutkan data cara 1}
\NormalTok{?arrange}
\NormalTok{mydata <-}\StringTok{ }\NormalTok{mydata }\OperatorTok
\StringTok{  }\KeywordTok{arrange}\NormalTok{(country)}
\NormalTok{mydata}
\end{Highlighting}
\end{Shaded}

\begin{verbatim}
## # A tibble: 6 x 4
##   country     year   kasus population
##   <chr>       <chr>  <int>      <int>
## 1 Afghanistan 1999     745   19987071
## 2 Afghanistan 2000    2666   20595360
## 3 Brazil      1999   37737  172006362
## 4 Brazil      2000   80488  174504898
## 5 China       1999  212258 1272915272
## 6 China       2000  213766 1280428583
\end{verbatim}

\begin{Shaded}
\begin{Highlighting}[]
\CommentTok{#mengurutkan data cara 2}
\NormalTok{mydata <-}\StringTok{ }\KeywordTok{arrange}\NormalTok{(mydata, country)}
\NormalTok{mydata}
\end{Highlighting}
\end{Shaded}

\begin{verbatim}
## # A tibble: 6 x 4
##   country     year   kasus population
##   <chr>       <chr>  <int>      <int>
## 1 Afghanistan 1999     745   19987071
## 2 Afghanistan 2000    2666   20595360
## 3 Brazil      1999   37737  172006362
## 4 Brazil      2000   80488  174504898
## 5 China       1999  212258 1272915272
## 6 China       2000  213766 1280428583
\end{verbatim}

\begin{Shaded}
\begin{Highlighting}[]
\CommentTok{#cek kesamaan 2 tabel}
\KeywordTok{identical}\NormalTok{(table1,mydata) }\CommentTok{#menghasilkan FALSE, artinya tidak identik dengan tabel 1 karena tipe data year berbeda.}
\end{Highlighting}
\end{Shaded}

\begin{verbatim}
## [1] FALSE
\end{verbatim}

\begin{Shaded}
\begin{Highlighting}[]
\CommentTok{#perlu kembali lagi kebagian gather, ditambahkan sintax convert = TRUE}

\CommentTok{#cara 2 pakai intermediate object}
\NormalTok{table4a_tidy}
\end{Highlighting}
\end{Shaded}

\begin{verbatim}
## # A tibble: 6 x 3
##   country     year   kasus
##   <chr>       <chr>  <int>
## 1 Afghanistan 1999     745
## 2 Brazil      1999   37737
## 3 China       1999  212258
## 4 Afghanistan 2000    2666
## 5 Brazil      2000   80488
## 6 China       2000  213766
\end{verbatim}

\begin{Shaded}
\begin{Highlighting}[]
\NormalTok{table4a_tidyy <-}\StringTok{ }\KeywordTok{gather}\NormalTok{(table4a, }\DataTypeTok{key =} \StringTok{"year"}\NormalTok{, }\DataTypeTok{value =} \StringTok{"kasus"}\NormalTok{, }\DecValTok{2}\OperatorTok{:}\DecValTok{3}\NormalTok{, }\DataTypeTok{convert =} \OtherTok{TRUE}\NormalTok{) }
\NormalTok{table4a_tidyy}
\end{Highlighting}
\end{Shaded}

\begin{verbatim}
## # A tibble: 6 x 3
##   country      year  kasus
##   <chr>       <int>  <int>
## 1 Afghanistan  1999    745
## 2 Brazil       1999  37737
## 3 China        1999 212258
## 4 Afghanistan  2000   2666
## 5 Brazil       2000  80488
## 6 China        2000 213766
\end{verbatim}

\begin{Shaded}
\begin{Highlighting}[]
\CommentTok{#cara 2 pakai Intermediate object}
\NormalTok{table4b_tidy}
\end{Highlighting}
\end{Shaded}

\begin{verbatim}
## # A tibble: 6 x 3
##   country     year  population
##   <chr>       <chr>      <int>
## 1 Afghanistan 1999    19987071
## 2 Brazil      1999   172006362
## 3 China       1999  1272915272
## 4 Afghanistan 2000    20595360
## 5 Brazil      2000   174504898
## 6 China       2000  1280428583
\end{verbatim}

\begin{Shaded}
\begin{Highlighting}[]
\NormalTok{table4b_tidyy <-}\StringTok{ }\KeywordTok{gather}\NormalTok{(table4b, }\DataTypeTok{key =}\StringTok{"year"}\NormalTok{, }\DataTypeTok{value =}\StringTok{"population"}\NormalTok{, }\DecValTok{2}\OperatorTok{:}\DecValTok{3}\NormalTok{, }\DataTypeTok{convert =} \OtherTok{TRUE}\NormalTok{)}
\NormalTok{table4b_tidyy}
\end{Highlighting}
\end{Shaded}

\begin{verbatim}
## # A tibble: 6 x 3
##   country      year population
##   <chr>       <int>      <int>
## 1 Afghanistan  1999   19987071
## 2 Brazil       1999  172006362
## 3 China        1999 1272915272
## 4 Afghanistan  2000   20595360
## 5 Brazil       2000  174504898
## 6 China        2000 1280428583
\end{verbatim}

\begin{Shaded}
\begin{Highlighting}[]
\CommentTok{#join lagi untuk tabel yy}
\NormalTok{mydata <-}\StringTok{ }\KeywordTok{left_join}\NormalTok{(table4a_tidyy, table4b_tidyy, }\DataTypeTok{by =} \KeywordTok{c}\NormalTok{(}\StringTok{"country"}\NormalTok{, }\StringTok{"year"}\NormalTok{))}
\NormalTok{mydata}
\end{Highlighting}
\end{Shaded}

\begin{verbatim}
## # A tibble: 6 x 4
##   country      year  kasus population
##   <chr>       <int>  <int>      <int>
## 1 Afghanistan  1999    745   19987071
## 2 Brazil       1999  37737  172006362
## 3 China        1999 212258 1272915272
## 4 Afghanistan  2000   2666   20595360
## 5 Brazil       2000  80488  174504898
## 6 China        2000 213766 1280428583
\end{verbatim}

\begin{Shaded}
\begin{Highlighting}[]
\NormalTok{mydata <-}\StringTok{ }\KeywordTok{arrange}\NormalTok{(mydata, country)}
\NormalTok{mydata}
\end{Highlighting}
\end{Shaded}

\begin{verbatim}
## # A tibble: 6 x 4
##   country      year  kasus population
##   <chr>       <int>  <int>      <int>
## 1 Afghanistan  1999    745   19987071
## 2 Afghanistan  2000   2666   20595360
## 3 Brazil       1999  37737  172006362
## 4 Brazil       2000  80488  174504898
## 5 China        1999 212258 1272915272
## 6 China        2000 213766 1280428583
\end{verbatim}

\begin{Shaded}
\begin{Highlighting}[]
\KeywordTok{identical}\NormalTok{(table1,mydata)}
\end{Highlighting}
\end{Shaded}

\begin{verbatim}
## [1] FALSE
\end{verbatim}

\hypertarget{data-wrangling}{%
\subsection{Data Wrangling}\label{data-wrangling}}

Dataset mydata tersebut merupakan subset dataset Tubercolusis yang
diolah dari data `who' dan `population' (dari paket \texttt{tidyr}).
Lihatlah ringkasan kedua tersebut dengan menggunakan \texttt{glimpse}!

\begin{Shaded}
\begin{Highlighting}[]
\CommentTok{#___(who)}
\CommentTok{#___}
\KeywordTok{glimpse}\NormalTok{(who)}
\end{Highlighting}
\end{Shaded}

\begin{verbatim}
## Observations: 7,240
## Variables: 60
## $ country      <chr> "Afghanistan", "Afghanistan", "Afghanistan", "Afg...
## $ iso2         <chr> "AF", "AF", "AF", "AF", "AF", "AF", "AF", "AF", "...
## $ iso3         <chr> "AFG", "AFG", "AFG", "AFG", "AFG", "AFG", "AFG", ...
## $ year         <int> 1980, 1981, 1982, 1983, 1984, 1985, 1986, 1987, 1...
## $ new_sp_m014  <int> NA, NA, NA, NA, NA, NA, NA, NA, NA, NA, NA, NA, N...
## $ new_sp_m1524 <int> NA, NA, NA, NA, NA, NA, NA, NA, NA, NA, NA, NA, N...
## $ new_sp_m2534 <int> NA, NA, NA, NA, NA, NA, NA, NA, NA, NA, NA, NA, N...
## $ new_sp_m3544 <int> NA, NA, NA, NA, NA, NA, NA, NA, NA, NA, NA, NA, N...
## $ new_sp_m4554 <int> NA, NA, NA, NA, NA, NA, NA, NA, NA, NA, NA, NA, N...
## $ new_sp_m5564 <int> NA, NA, NA, NA, NA, NA, NA, NA, NA, NA, NA, NA, N...
## $ new_sp_m65   <int> NA, NA, NA, NA, NA, NA, NA, NA, NA, NA, NA, NA, N...
## $ new_sp_f014  <int> NA, NA, NA, NA, NA, NA, NA, NA, NA, NA, NA, NA, N...
## $ new_sp_f1524 <int> NA, NA, NA, NA, NA, NA, NA, NA, NA, NA, NA, NA, N...
## $ new_sp_f2534 <int> NA, NA, NA, NA, NA, NA, NA, NA, NA, NA, NA, NA, N...
## $ new_sp_f3544 <int> NA, NA, NA, NA, NA, NA, NA, NA, NA, NA, NA, NA, N...
## $ new_sp_f4554 <int> NA, NA, NA, NA, NA, NA, NA, NA, NA, NA, NA, NA, N...
## $ new_sp_f5564 <int> NA, NA, NA, NA, NA, NA, NA, NA, NA, NA, NA, NA, N...
## $ new_sp_f65   <int> NA, NA, NA, NA, NA, NA, NA, NA, NA, NA, NA, NA, N...
## $ new_sn_m014  <int> NA, NA, NA, NA, NA, NA, NA, NA, NA, NA, NA, NA, N...
## $ new_sn_m1524 <int> NA, NA, NA, NA, NA, NA, NA, NA, NA, NA, NA, NA, N...
## $ new_sn_m2534 <int> NA, NA, NA, NA, NA, NA, NA, NA, NA, NA, NA, NA, N...
## $ new_sn_m3544 <int> NA, NA, NA, NA, NA, NA, NA, NA, NA, NA, NA, NA, N...
## $ new_sn_m4554 <int> NA, NA, NA, NA, NA, NA, NA, NA, NA, NA, NA, NA, N...
## $ new_sn_m5564 <int> NA, NA, NA, NA, NA, NA, NA, NA, NA, NA, NA, NA, N...
## $ new_sn_m65   <int> NA, NA, NA, NA, NA, NA, NA, NA, NA, NA, NA, NA, N...
## $ new_sn_f014  <int> NA, NA, NA, NA, NA, NA, NA, NA, NA, NA, NA, NA, N...
## $ new_sn_f1524 <int> NA, NA, NA, NA, NA, NA, NA, NA, NA, NA, NA, NA, N...
## $ new_sn_f2534 <int> NA, NA, NA, NA, NA, NA, NA, NA, NA, NA, NA, NA, N...
## $ new_sn_f3544 <int> NA, NA, NA, NA, NA, NA, NA, NA, NA, NA, NA, NA, N...
## $ new_sn_f4554 <int> NA, NA, NA, NA, NA, NA, NA, NA, NA, NA, NA, NA, N...
## $ new_sn_f5564 <int> NA, NA, NA, NA, NA, NA, NA, NA, NA, NA, NA, NA, N...
## $ new_sn_f65   <int> NA, NA, NA, NA, NA, NA, NA, NA, NA, NA, NA, NA, N...
## $ new_ep_m014  <int> NA, NA, NA, NA, NA, NA, NA, NA, NA, NA, NA, NA, N...
## $ new_ep_m1524 <int> NA, NA, NA, NA, NA, NA, NA, NA, NA, NA, NA, NA, N...
## $ new_ep_m2534 <int> NA, NA, NA, NA, NA, NA, NA, NA, NA, NA, NA, NA, N...
## $ new_ep_m3544 <int> NA, NA, NA, NA, NA, NA, NA, NA, NA, NA, NA, NA, N...
## $ new_ep_m4554 <int> NA, NA, NA, NA, NA, NA, NA, NA, NA, NA, NA, NA, N...
## $ new_ep_m5564 <int> NA, NA, NA, NA, NA, NA, NA, NA, NA, NA, NA, NA, N...
## $ new_ep_m65   <int> NA, NA, NA, NA, NA, NA, NA, NA, NA, NA, NA, NA, N...
## $ new_ep_f014  <int> NA, NA, NA, NA, NA, NA, NA, NA, NA, NA, NA, NA, N...
## $ new_ep_f1524 <int> NA, NA, NA, NA, NA, NA, NA, NA, NA, NA, NA, NA, N...
## $ new_ep_f2534 <int> NA, NA, NA, NA, NA, NA, NA, NA, NA, NA, NA, NA, N...
## $ new_ep_f3544 <int> NA, NA, NA, NA, NA, NA, NA, NA, NA, NA, NA, NA, N...
## $ new_ep_f4554 <int> NA, NA, NA, NA, NA, NA, NA, NA, NA, NA, NA, NA, N...
## $ new_ep_f5564 <int> NA, NA, NA, NA, NA, NA, NA, NA, NA, NA, NA, NA, N...
## $ new_ep_f65   <int> NA, NA, NA, NA, NA, NA, NA, NA, NA, NA, NA, NA, N...
## $ newrel_m014  <int> NA, NA, NA, NA, NA, NA, NA, NA, NA, NA, NA, NA, N...
## $ newrel_m1524 <int> NA, NA, NA, NA, NA, NA, NA, NA, NA, NA, NA, NA, N...
## $ newrel_m2534 <int> NA, NA, NA, NA, NA, NA, NA, NA, NA, NA, NA, NA, N...
## $ newrel_m3544 <int> NA, NA, NA, NA, NA, NA, NA, NA, NA, NA, NA, NA, N...
## $ newrel_m4554 <int> NA, NA, NA, NA, NA, NA, NA, NA, NA, NA, NA, NA, N...
## $ newrel_m5564 <int> NA, NA, NA, NA, NA, NA, NA, NA, NA, NA, NA, NA, N...
## $ newrel_m65   <int> NA, NA, NA, NA, NA, NA, NA, NA, NA, NA, NA, NA, N...
## $ newrel_f014  <int> NA, NA, NA, NA, NA, NA, NA, NA, NA, NA, NA, NA, N...
## $ newrel_f1524 <int> NA, NA, NA, NA, NA, NA, NA, NA, NA, NA, NA, NA, N...
## $ newrel_f2534 <int> NA, NA, NA, NA, NA, NA, NA, NA, NA, NA, NA, NA, N...
## $ newrel_f3544 <int> NA, NA, NA, NA, NA, NA, NA, NA, NA, NA, NA, NA, N...
## $ newrel_f4554 <int> NA, NA, NA, NA, NA, NA, NA, NA, NA, NA, NA, NA, N...
## $ newrel_f5564 <int> NA, NA, NA, NA, NA, NA, NA, NA, NA, NA, NA, NA, N...
## $ newrel_f65   <int> NA, NA, NA, NA, NA, NA, NA, NA, NA, NA, NA, NA, N...
\end{verbatim}

\begin{Shaded}
\begin{Highlighting}[]
\KeywordTok{glimpse}\NormalTok{(population)}
\end{Highlighting}
\end{Shaded}

\begin{verbatim}
## Observations: 4,060
## Variables: 3
## $ country    <chr> "Afghanistan", "Afghanistan", "Afghanistan", "Afgha...
## $ year       <int> 1995, 1996, 1997, 1998, 1999, 2000, 2001, 2002, 200...
## $ population <int> 17586073, 18415307, 19021226, 19496836, 19987071, 2...
\end{verbatim}

Sekarang kita akan membuat versi utuh dari dataset `mydata' dengan
menggunakan data seluruh negara pada dataset `who' sebagai berikut:

\begin{Shaded}
\begin{Highlighting}[]
\CommentTok{# Menjalankan fungsi satu per satu}
\NormalTok{tb1 <-}\StringTok{ }\KeywordTok{gather}\NormalTok{(who, }\DataTypeTok{key =} \StringTok{"key"}\NormalTok{, }\DataTypeTok{value =} \StringTok{"case"}\NormalTok{, new_sp_m014}\OperatorTok{:}\NormalTok{newrel_f65)}
\KeywordTok{View}\NormalTok{(tb1)}
\CommentTok{#cara lain untuk menampilkan langsung ke layar, tambah kurung diawal dan akhir}
\NormalTok{(tb1 <-}\StringTok{ }\KeywordTok{gather}\NormalTok{(who, }\DataTypeTok{key =} \StringTok{"key"}\NormalTok{, }\DataTypeTok{value =} \StringTok{"case"}\NormalTok{, new_sp_m014}\OperatorTok{:}\NormalTok{newrel_f65))}
\end{Highlighting}
\end{Shaded}

\begin{verbatim}
## # A tibble: 405,440 x 6
##    country     iso2  iso3   year key          case
##    <chr>       <chr> <chr> <int> <chr>       <int>
##  1 Afghanistan AF    AFG    1980 new_sp_m014    NA
##  2 Afghanistan AF    AFG    1981 new_sp_m014    NA
##  3 Afghanistan AF    AFG    1982 new_sp_m014    NA
##  4 Afghanistan AF    AFG    1983 new_sp_m014    NA
##  5 Afghanistan AF    AFG    1984 new_sp_m014    NA
##  6 Afghanistan AF    AFG    1985 new_sp_m014    NA
##  7 Afghanistan AF    AFG    1986 new_sp_m014    NA
##  8 Afghanistan AF    AFG    1987 new_sp_m014    NA
##  9 Afghanistan AF    AFG    1988 new_sp_m014    NA
## 10 Afghanistan AF    AFG    1989 new_sp_m014    NA
## # ... with 405,430 more rows
\end{verbatim}

\begin{Shaded}
\begin{Highlighting}[]
\NormalTok{tb2 <-}\StringTok{ }\KeywordTok{select}\NormalTok{(tb1, country, year, case)}
\KeywordTok{View}\NormalTok{(tb2)}
\NormalTok{tb3 <-}\StringTok{ }\KeywordTok{group_by}\NormalTok{(tb2, country, year)}
\KeywordTok{View}\NormalTok{(tb3)}
\NormalTok{tb4 <-}\StringTok{ }\KeywordTok{summarise}\NormalTok{(tb3, }\DataTypeTok{cases =} \KeywordTok{sum}\NormalTok{(case, }\DataTypeTok{na.rm =} \OtherTok{TRUE}\NormalTok{)) }\CommentTok{#mengabaikan nilai N/A}
\KeywordTok{View}\NormalTok{(tb4)}
\NormalTok{(tb5 <-}\StringTok{ }\KeywordTok{ungroup}\NormalTok{(tb4))}
\end{Highlighting}
\end{Shaded}

\begin{verbatim}
## # A tibble: 7,240 x 3
##    country      year cases
##    <chr>       <int> <int>
##  1 Afghanistan  1980     0
##  2 Afghanistan  1981     0
##  3 Afghanistan  1982     0
##  4 Afghanistan  1983     0
##  5 Afghanistan  1984     0
##  6 Afghanistan  1985     0
##  7 Afghanistan  1986     0
##  8 Afghanistan  1987     0
##  9 Afghanistan  1988     0
## 10 Afghanistan  1989     0
## # ... with 7,230 more rows
\end{verbatim}

\begin{Shaded}
\begin{Highlighting}[]
\NormalTok{(tb6 <-}\StringTok{ }\KeywordTok{left_join}\NormalTok{(tb5, population))}
\end{Highlighting}
\end{Shaded}

\begin{verbatim}
## Joining, by = c("country", "year")
\end{verbatim}

\begin{verbatim}
## # A tibble: 7,240 x 4
##    country      year cases population
##    <chr>       <int> <int>      <int>
##  1 Afghanistan  1980     0         NA
##  2 Afghanistan  1981     0         NA
##  3 Afghanistan  1982     0         NA
##  4 Afghanistan  1983     0         NA
##  5 Afghanistan  1984     0         NA
##  6 Afghanistan  1985     0         NA
##  7 Afghanistan  1986     0         NA
##  8 Afghanistan  1987     0         NA
##  9 Afghanistan  1988     0         NA
## 10 Afghanistan  1989     0         NA
## # ... with 7,230 more rows
\end{verbatim}

\begin{Shaded}
\begin{Highlighting}[]
\NormalTok{(tb7 <-}\StringTok{ }\KeywordTok{filter}\NormalTok{(tb6, }\OperatorTok{!}\KeywordTok{is.na}\NormalTok{(population)))}
\end{Highlighting}
\end{Shaded}

\begin{verbatim}
## # A tibble: 4,037 x 4
##    country      year cases population
##    <chr>       <int> <int>      <int>
##  1 Afghanistan  1995     0   17586073
##  2 Afghanistan  1996     0   18415307
##  3 Afghanistan  1997   128   19021226
##  4 Afghanistan  1998  1778   19496836
##  5 Afghanistan  1999   745   19987071
##  6 Afghanistan  2000  2666   20595360
##  7 Afghanistan  2001  4639   21347782
##  8 Afghanistan  2002  6509   22202806
##  9 Afghanistan  2003  6528   23116142
## 10 Afghanistan  2004  8245   24018682
## # ... with 4,027 more rows
\end{verbatim}

\begin{Shaded}
\begin{Highlighting}[]
\NormalTok{(tb8 <-}\StringTok{ }\KeywordTok{mutate}\NormalTok{(tb7, }\DataTypeTok{proportion =} \DecValTok{100}\OperatorTok{*}\NormalTok{cases}\OperatorTok{/}\NormalTok{population))}
\end{Highlighting}
\end{Shaded}

\begin{verbatim}
## # A tibble: 4,037 x 5
##    country      year cases population proportion
##    <chr>       <int> <int>      <int>      <dbl>
##  1 Afghanistan  1995     0   17586073   0       
##  2 Afghanistan  1996     0   18415307   0       
##  3 Afghanistan  1997   128   19021226   0.000673
##  4 Afghanistan  1998  1778   19496836   0.00912 
##  5 Afghanistan  1999   745   19987071   0.00373 
##  6 Afghanistan  2000  2666   20595360   0.0129  
##  7 Afghanistan  2001  4639   21347782   0.0217  
##  8 Afghanistan  2002  6509   22202806   0.0293  
##  9 Afghanistan  2003  6528   23116142   0.0282  
## 10 Afghanistan  2004  8245   24018682   0.0343  
## # ... with 4,027 more rows
\end{verbatim}

\begin{Shaded}
\begin{Highlighting}[]
\CommentTok{# Syntax menggunakan pipe %>%}

\NormalTok{tb_all <-}\StringTok{ }
\StringTok{  }\NormalTok{who }\OperatorTok\StringTok{ }
\StringTok{  }\KeywordTok{gather}\NormalTok{(}\DataTypeTok{key =} \StringTok{"key"}\NormalTok{, }\DataTypeTok{value =} \StringTok{"case"}\NormalTok{, new_sp_m014}\OperatorTok{:}\NormalTok{newrel_f65) }\OperatorTok\StringTok{ }
\StringTok{  }\KeywordTok{select}\NormalTok{(country, year, case) }\OperatorTok\StringTok{ }
\StringTok{  }\KeywordTok{group_by}\NormalTok{(country, year) }\OperatorTok\StringTok{ }
\StringTok{  }\KeywordTok{summarise}\NormalTok{(}\DataTypeTok{cases =} \KeywordTok{sum}\NormalTok{(case, }\DataTypeTok{na.rm =} \OtherTok{TRUE}\NormalTok{)) }\OperatorTok\StringTok{ }
\StringTok{  }\KeywordTok{ungroup}\NormalTok{() }\OperatorTok\StringTok{ }
\StringTok{  }\KeywordTok{left_join}\NormalTok{(population, }\DataTypeTok{by =} \KeywordTok{c}\NormalTok{(}\StringTok{"country"}\NormalTok{, }\StringTok{"year"}\NormalTok{)) }\OperatorTok\StringTok{ }
\StringTok{  }\KeywordTok{filter}\NormalTok{(}\OperatorTok{!}\KeywordTok{is.na}\NormalTok{(population)) }\OperatorTok\StringTok{ }
\StringTok{  }\KeywordTok{mutate}\NormalTok{(}\DataTypeTok{proportion =} \DecValTok{100}\OperatorTok{*}\NormalTok{cases}\OperatorTok{/}\NormalTok{population)}
\end{Highlighting}
\end{Shaded}

Dapatkah Anda membuat ringkasan apa saja hal apa saja yang dilakukan
pada proses \emph{data wrangling} diatas? (Petunjuk:
\texttt{?nama\_fungsi})

\begin{enumerate}
\def\labelenumi{\arabic{enumi}.}
\tightlist
\item
  \ldots{}
\item
  \ldots{}
\item
  \ldots{}
\item
  \ldots{}
\item
  \ldots{}
\item
  \ldots{}
\item
  \ldots{}
\item
  \ldots{}
\end{enumerate}

Cek apakah dataset `tb8' sama dengan dataset `tb\_all'! Menurut Anda,
cara penulisan \emph{syntax} manakah yang lebih mudah digunakan dan
dipahami?

\begin{Shaded}
\begin{Highlighting}[]
\KeywordTok{identical}\NormalTok{(tb8, tb_all)}
\end{Highlighting}
\end{Shaded}

\begin{verbatim}
## [1] TRUE
\end{verbatim}


\end{document}
